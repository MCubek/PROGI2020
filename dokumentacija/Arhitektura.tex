\chapter{Arhitektura i dizajn sustava}
		

	\textnormal{Pri procesu izbora arhitekture naše web aplikacije fokusirali smo se na klijent-server strukturu uz mogućnost odvojenih podsustava. U takvoj strukturi web poslužitelj osnova je rada web aplikacije. Njegova primarna zadaća je komunikacija klijenta s aplikacijom putem HTTP (\textit{engl. Hyper Text Transfer Protocol}) protokola. Korisnik šalje zahtjeve poslužitelju putem web preglednika. Web preglednik je program koji korisniku omogućuje pregled web-stranica i multimedijalnih sadržaja vezanih uz njih tako što stranicu pisanu u kodu interpretira na način da je razumljiva korisniku. Na temelju poslanih zahtjeva web aplikacija pristupa bazi podataka te vraća odgovor putem poslužitelja u obliku HTML dokumenta koji je vidljiv u pregledniku.}

	\begin{figure}[H]
		\centering
		\includegraphics[scale=0.7]{slike/Arhitektura} \\
		\caption{ Organizacija sustava}
		\label{fig:arhitektura}
	\end{figure}

	\textnormal{Na temelju navedenih glavnih značajki i nadalje razrađenih obilježja odlučili smo se arhitekturu sustava bazirati na arhitekturnom stilu model-pogled-upravitelj (\textit{engl. Model-View-Controller, MVC}). Kod MVC stila, korisničko sučelje je odvojeno od ostatka sustava što nam omogućuje raspodjelu u odvojene timove tijekom rada na projektu. Mnogi programski okviri su izrađeni s MVC arhitekturom na umu, što omogućuje brži i jednostavniji razvoj aplikacije. Razinska kohezija elemenata postiže se kroz tri sloja, jednog na strani klijenta koji se naziva pogled (\textit{engl. view}) i dva na poslužiteljskoj strani, koji se nazivaju upravitelj (\textit{engl. controller}) i model (\textit{engl. model}), što nam uz povećanu koheziju i smanjivanje međuovisnosti omogućuje i fleksibilnost. Također smo izabrali ovaj arhitekturni stil zbog njegove mogućnosti jednostavne ponovne uporabe i nadogradnje pojedinačnih dijelova sustava, pošto se dodavanjem novih modela ili pogleda može brzo i jednostavno promijeniti aplikaciju. Time se pozitivno utječe i na planiranje zastare aplikacije, uz korištenje dobro poznatih tehnologija i jezika koji će biti detaljnije razrađeni u nastavku. Podjelom na navedene podsustave pojednostavili smo i testiranje tijekom izrade aplikacije jer se razni dijelovi sustava mogu neovisno testirati. Izborom web aplikacije pobrinuli smo se i za prenosivost, pošto se aplikaciji može pristupiti s velikog broja uređaja putem različitih web preglednika.}\\
	
	\textnormal{Negativna karakteristika MVC arhitekture može biti njegova glomaznost u pogledu velike količine datoteka što može usporavati sustav, no to ne smatramo problemom za ovaj projekt pošto je manjeg opsega.}\\
	
	\begin{figure}[H]
		\centering
		\includegraphics[scale=0.5]{slike/MVC} \\
		\caption{ MVC Arhitektura}
		\label{fig:arhitekturaMCV}
	\end{figure}

	\textnormal{U skladu s odabranom arhitekturom, aplikacija će biti razvijena u tri velike nezavisne komponente: pogled kroz \textit{front-end} sustav, te upravitelj i model kroz \textit{back-end} sustav i bazu podataka.}\\
	
	\textnormal{\textit{Front-end} predstavlja korisničko sučelje s kojim korisnik ima direktnu interakciju. Implementiran je pomoću programskog okvira Angular koristeći JavaScript, HTML i CSS. Za Angular smo se odlučili jer omogućava dizajniranje fleksibilnih komponenti visoke razine ponovne upotrebljivosti uz veliku responzivnost i interaktivnost. Uloge Angular servisa u ovoj aplikaciji su dostaviti sve potrebne podatke i skripte web poslužitelju korisnika, komunikacija s back end servisom i izvođenje vlastitih procesa koji su nužni za pravilan rad korisničkog sučelja. Razdvajanjem \textit{front-end} i \textit{back-end} servisa omogućavamo fokus na razvoj odličnog korisničkog sučelja što je bitan faktor korisnicima pri odabiru naše igre kao jedne od mnogo dostupnih.}\\
	
	\textnormal{\textit{Back-end} predstavlja servis koji upravlja \textit{front-end} servisom i dostavlja podatke korisniku te je zato centralna točka GeoFighter aplikacije. Implementiran je pomoću programskog okvira SpringBoot koristeći programski jezik Java. Omogućava pristup bazi podataka, komunikaciju HTTP zahtjevima te sadržava logiku i funkcionalnosti GeoFighter igre. Struktura \textit{back-end} servisa se fokusira na smanjivanje međuovisnosti sadržaja i izbjegava opću međuovisnost. Razdvajanjem \textit{front-end} i \textit{back-end} servisa smanjujemo opterećenost \textit{back-end} servisa}\\
	
	\textnormal{Baza podataka je trajno spremište svih podataka koje GeoFighter aplikacija treba sadržavati. Ona komunicira isključivo s \textit{back-end} servisom. Baza podataka je relacijskog tipa te je implementirana koristeći Postgres. Postgres baza podataka je odličan izbor za baze podataka male do srednje veličine te je vrlo fleksibilan izbor za zahtjeve koje ova aplikacija izvršava.}\\
	
	\begin{figure}[H]
		\centering
		\includegraphics[scale=0.5]{slike/strukturaGF} \\
		\caption{Skica arhitekture sustava}
		\label{fig:arhitekturaGF}
	\end{figure}
	
	
				
		\section{Baza podataka}
			
			
		\textnormal{Za potrebe našeg sustava koristiti ćemo relacijsku bazu podataka koja svojom strukturom 	olakšava modeliranje stvarnog svijeta. Gradivna jedinka baze je relacija, odnosno tablica koja je definirana svojim imenom i skupom atributa, dok su odnosi između relacija definirani vezama. Zadaća baze podataka je brza i jednostavna pohrana, izmjena i dohvat podataka za daljnju obradu.\\Baza podataka ove aplikacije sastoji se od slijedećih tablica:}
		
		\begin{packed_item}
			\item \textnormal{Users}
			\item \textnormal{Tokens}
			\item \textnormal{Refresh\textunderscore token}
			\item \textnormal{Images}
			\item \textnormal{Roles}
			\item \textnormal{Privileges}
			\item \textnormal{Roles\textunderscore privileges}
			\item \textnormal{Location\textunderscore cards}
			\item \textnormal{User\textunderscore card}
			\item \textnormal{Fights}
			\item \textnormal{User\textunderscore  card\textunderscore fight}
		\end{packed_item}
		
			\subsection{Opis tablica}
			

				\textnormal{\textbf{Users} \quad Ovaj entitet sadržava sve važne informacije o korisniku aplikacije koje bilježi atributima navedenim u tablici. U vezi je \textit{One-to-Many} s entitetima Images, Tokens, Fights, Location\textunderscore cards i User\textunderscore card preko atributa user\textunderscore id. Također je u vezi \textit{Many-to-One} s entitetom Roles preko atributa role\textunderscore id.}
				
				\begin{longtabu} to \textwidth {|X[6, 3]|X[6, l]|X[20, l]|}
					
					\hline \multicolumn{3}{|c|}{\textbf{users}}	 \\[3pt] \hline
					\endfirsthead
					
					\hline \multicolumn{3}{|c|}{\textbf{users}}	 \\[3pt] \hline
					\endhead
					
					\hline 
					\endlastfoot
					
					\cellcolor{LightGreen}user\textunderscore id & BIGSERIAL	&  	jedinstveni identifikator korisnika	\\ \hline
					email & VARCHAR & email adresa korisnika \\ \hline
					username & VARCHAR & korisničko ime \\ \hline
					password & VARCHAR & lozinka korisnika \\ \hline
					created\textunderscore time & TIMESTAMP & trenutak stvaranja korisničkog računa  \\ \hline 
					current\textunderscore location & BYTEA	&  	trenutna lokacija korisnika	\\ \hline 
					elo\textunderscore score & INT4 & rang korisnika prema elo sustavu \\ \hline
					enabled & BOOLEAN & \\ \hline
					losses & INT4 & broj izgubljenih igara \\ \hline
					wins & INT4 & broj pobijeđenih igara \\ \hline
					online & BOOLEAN & zastavica prisutnosti korisnika \\ \hline
					forced\textunderscore timeout\textunderscore end & TIMESTAMP & trenutak mogućeg povratka u igru u slučaju privremenog isključenja \\ \hline
					cartographer\textunderscore status	& INT4 &   	\\ \hline 
					iban & VARCHAR & IBAN bankovnog računa kartografa \\ \hline			
					
					\cellcolor{LightBlue} role\textunderscore id	& INT8 &  jedinstveni identifikator uloge (roles.role\textunderscore id)	\\ \hline 
					
					
				\end{longtabu}
			
				
				\textnormal{\textbf{Tokens} \quad Ovaj entitet sadržava sve važne informacije o tokenima koji se koriste za autentifikaciju korisnika. Atributi koje posjeduje su jedinstveni identifikator tokena, njegova vrijednost, datum isticanja te oznaku korisnika kojem pripada. U vezi je \textit{Many-to-One} s entitetom Users preko atributa "user\textunderscore user\textunderscore id". Također je u vezi \textit{One-to-One} s entitetom Refresh\textunderscore token preko atributa id.}
			
			
			
				\begin{longtabu} to \textwidth {|X[6, l]|X[6, l]|X[20, l]|}
					
					\hline \multicolumn{3}{|c|}{\textbf{tokens}}	 \\[3pt] \hline
					\endfirsthead
					
					\hline \multicolumn{3}{|c|}{\textbf{tokens}}	 \\[3pt] \hline
					\endhead
					
					\hline 
					\endlastfoot
					
					\cellcolor{LightGreen}id & BIGSERIAL	&  	jedinstveni identifikator tokena 	\\ \hline
					expiry\textunderscore date	& TIMESTAMP &   datum isticanja tokena	\\ \hline 
					token & VARCHAR &  vrijednost tokena \\ \hline  
					\cellcolor{LightBlue} "user\textunderscore user\textunderscore id"	& INT8 &  jedinstvena oznaka korisnika (users.user\textunderscore id) 	\\ \hline 
					
					
				\end{longtabu}
			
				\textnormal{\textbf{Refresh\textunderscore token} \quad Ovaj entitet sadržava sve važne informacije o obnavljanju tokena koji služe za autentifikaciju korisnika. Atributi koje posjeduje su jedinstveni identifikator tokena, njegova nova vrijednost te datum stvaranja. U vezi je \textit{One-to-One} s entitetom Tokens preko atributa id.}
				
			\begin{longtabu} to \textwidth {|X[6, l]|X[6, l]|X[20, l]|}
				
				\hline \multicolumn{3}{|c|}{\textbf{refresh\textunderscore token}}	 \\[3pt] \hline
				\endfirsthead
				
				\hline \multicolumn{3}{|c|}{\textbf{refresh\textunderscore token}}	 \\[3pt] \hline
				\endhead
				
				\hline 
				\endlastfoot
				
				\cellcolor{LightGreen}id & BIGSERIAL &  	jedinstveni identifikator tokena (tokens.id) 	\\ \hline
				created\textunderscore date	& TIMESTAMP &   datum stvaranja tokena	\\ \hline 
				token & VARCHAR & vrijednost tokena  \\ \hline 
				
				
				
			\end{longtabu}
		
			\textnormal{\textbf{Images} \quad Ovaj entitet sadržava sve informacije o fotografijama koje je potrebno priložiti pri registraciji korisnika. Atributi koje posjeduje su jedinstveni identifikator fotografije, njen digitalni zapis, vrstu fotografije koja se odnosi na fotgrafiju korisnika ili fotografiju osobne iskaznice te jedinstvenu oznaku korisnika kojem pripada. U vezi je \textit{Many-to-One} s entitetom Users preko atributa "user\textunderscore user\textunderscore id". }
		
			\begin{longtabu} to \textwidth {|X[6, l]|X[6, l]|X[20, l]|}
				
				\hline \multicolumn{3}{|c|}{\textbf{images}}	 \\[3pt] \hline
				\endfirsthead
				
				\hline \multicolumn{3}{|c|}{\textbf{images}}	 \\[3pt] \hline
				\endhead
				
				\hline 
				\endlastfoot
				
				\cellcolor{LightGreen}id & BIGSERIAL	&  	jedinstveni identifikator fotografije 	\\ \hline
				pic\textunderscore byte	& OID &  digitalni zapis fotografije 	\\ \hline 
				type & VARCHAR & vrsta fotografije  \\ \hline  
				\cellcolor{LightBlue} "user\textunderscore user\textunderscore id"	& INT8 &  jedinstvena oznaka korisnika (users.user\textunderscore id)   	\\ \hline 
				
				
			\end{longtabu}
		
			\textnormal{\textbf{Roles} \quad Ovaj entitet sadržava sve važne informacije vezane uz ulogu koju određeni korisnik može imati, a one su administrator, kartograf ili igrač. Atributi koje posjeduje su jedinstveni identifikator uloge i njezin naziv. U vezi je \textit{One-to-Many} s entitetom Users preko atributa role\textunderscore id. Također je u vezi \textit{Many-to-Many} s entitetom Privileges preko spojne tablice roles\textunderscore privileges.}
		
			\begin{longtabu} to \textwidth {|X[6, l]|X[6, l]|X[20, l]|}
				
				\hline \multicolumn{3}{|c|}{\textbf{roles}}	 \\[3pt] \hline
				\endfirsthead
				
				\hline \multicolumn{3}{|c|}{\textbf{roles}}	 \\[3pt] \hline
				\endhead
				
				\hline 
				\endlastfoot
				
				\cellcolor{LightGreen}role\textunderscore id & INT8	&  	jedinstveni identifikator uloge 	\\ \hline
				name	& VARCHAR &   naziv uloge	\\ \hline 
							
				
			\end{longtabu}
		
			\textnormal{\textbf{Privileges} \quad Ovaj entitet sadržava sve važne informacije o povlasticama koje uloge mogu imati. Atributi koje posjeduje su jedinstveni identifikator povlastice i njen naziv. U vezi je \textit{Many-to-Many} s entitetom Roles preko spojne tablice roles\textunderscore privileges.}
		
			\begin{longtabu} to \textwidth {|X[6, l]|X[6, l]|X[20, l]|}
				
				\hline \multicolumn{3}{|c|}{\textbf{privileges}}	 \\[3pt] \hline
				\endfirsthead
				
				\hline \multicolumn{3}{|c|}{\textbf{privileges}}	 \\[3pt] \hline
				\endhead
				
				\hline 
				\endlastfoot
				
				\cellcolor{LightGreen}privilege\textunderscore id & INT8	&  	jedinstveni identifikator povlastice 	\\ \hline
				name	& VARCHAR &   naziv povlastice	\\ \hline 
								
				
			\end{longtabu}
		
			\textnormal{\textbf{Roles\textunderscore privileges} \quad Ova tablica nastaje zbog \textit{Many-to-Many} veze između entiteta Roles i Privileges. Sadrži njihove primarne ključeve te je time povezana s njima. Prikazuje koje uloge imaju koje povlastice.}
			
			\begin{longtabu} to \textwidth {|X[6, l]|X[6, l]|X[20, l]|}
				
				\hline \multicolumn{3}{|c|}{\textbf{roles\textunderscore privileges}}	 \\[3pt] \hline
				\endfirsthead
				
				\hline \multicolumn{3}{|c|}{\textbf{roles\textunderscore privileges}}	 \\[3pt] \hline
				\endhead
				
				\hline 
				\endlastfoot
				
				\cellcolor{LightGreen} role\textunderscore id & INT8 & jedinstveni identifikator uloge (roles.role\textunderscore id) \\ \hline
				\cellcolor{LightGreen}privilege\textunderscore id & INT8	&  	jedinstveni identifikator povlastice (privileges.privilege\textunderscore id) 	\\ \hline
				
						
				
			\end{longtabu}
		
			\textnormal{\textbf{Location\textunderscore cards} \quad Ovaj entitet sadržava sve važne informacije o karticama lokacije. Atributi koje posjeduje su navedeni u tablici. U vezi je \textit{Many-to-Many-to-Many} s entitetima Fights i Users preko spojne tablice User\textunderscore card\textunderscore fight. Također je u vezi \textit{Many-to-One} s entitetom Users preko atributa accepted\textunderscore by\textunderscore user \textunderscore id i atributa created\textunderscore by\textunderscore user \textunderscore id.}
		
			\begin{longtabu} to \textwidth {|X[6, 4]|X[6, l]|X[20, l]|}
				
				\hline \multicolumn{3}{|c|}{\textbf{location\textunderscore cards}}	 \\[3pt] \hline
				\endfirsthead
				
				\hline \multicolumn{3}{|c|}{\textbf{location\textunderscore cards}}	 \\[3pt] \hline
				\endhead
				
				\hline 
				\endlastfoot
				
				\cellcolor{LightGreen}lcd\textunderscore id & INT8	&  	jedinstveni identifikator kartice lokacije 	\\ \hline
				name	& VARCHAR &   naziv lokacije	\\ \hline 
				accepted & BOOLEAN & zastavica koja označava je li lokacija odobrena \\ \hline
				created\textunderscore at & TIMESTAMP & trenutak stvaranja kartice lokacije \\ \hline
				description & TEXT & opis lokacije \\ \hline
				enabled\textunderscore date & TIMESTAMP & trenutak odobravanja kartice lokacije \\ \hline
				location & BYTEA & stvarna lokacija koju označava kartica lokacije \\ \hline
				check\textunderscore needed & BOOLEAN & zastavica koja označava potrebu za dodatnom potvrdom kartice lokacije \\ \hline
				photo & OID & digitalni zapis fotografije lokacije \\ \hline
				\cellcolor{LightBlue}accepted\textunderscore by\textunderscore user \textunderscore id & INT8 & jedinstveni identifikator kartografa koji je odobrio karticu lokacije (users.user\textunderscore id) \\ \hline
				\cellcolor{LightBlue}created\textunderscore by\textunderscore user \textunderscore id & INT8 & jedinstveni identifikator igrača koji je prijavio karticu lokacije (users.user\textunderscore id) \\ \hline				
				
				
			\end{longtabu}
		
			\textnormal{\textbf{User\textunderscore card} \quad Ovaj entitet sadržava sve informacije o kartici koju posjeduje neki igrač. Atributi koje posjeduje su jedinstveni identifikator korisnika i kartice koju posjeduje, trenutak kada se kartica može ponovo koristit u borbi te koeficijent koji utječe na smanjenje broja bodova kartice. U vezi je \textit{Many-to-One} s entitetom Users preko atributa user\textunderscore id te s entitetom Location\textunderscore cards preko atribut lcd\textunderscore id.}
		
			\begin{longtabu} to \textwidth {|X[6, 4]|X[6, l]|X[20, l]|}
				
				\hline \multicolumn{3}{|c|}{\textbf{user\textunderscore card}}	 \\[3pt] \hline
				\endfirsthead
				
				\hline \multicolumn{3}{|c|}{\textbf{user\textunderscore card}}	 \\[3pt] \hline
				\endhead
				
				\hline 
				\endlastfoot
				
				\cellcolor{LightGreen}user\textunderscore id & INT8	&  	jedinstveni identifikator korisnika koji posjeduje karticu lokacije (users.user\textunderscore id) 	\\ \hline
				\cellcolor{LightGreen}lcd\textunderscore id & INT8	&  	jedinstveni identifikator kartice lokacije koju korisnik posjeduje (location\textunderscore cards.lcd\textunderscore id) 	\\ \hline
				cooldown\textunderscore end	& TIMESTAMP &   trenutak kada se kartica lokacije može ponovo koristiti u borbi	\\ \hline 
				cooldown\textunderscore multiplier & FLOAT8 & koeficijent prema kojem kartica lokacije gubi bodove \\ \hline
				
				
			\end{longtabu}
		
			\textnormal{\textbf{Fights\textunderscore cards} \quad Ovaj entitet sadržava sve važne informacije o borbama koje su se odvile. Atributi koje posjeduje su jedinstveni identifikator borbe, vrijeme početka borbe te igrača koji je pobijedio. navedeni u tablici. U vezi je \textit{Many-to-Many-to-Many} s entitetima Location\textunderscore cards i Users preko spojne tablice User\textunderscore card\textunderscore fight. Također je u vezi \textit{Many-to-One} s entitetom Users preko atributa winner\textunderscore user \textunderscore id.}
		
			\begin{longtabu} to \textwidth {|X[6, l]|X[6, l]|X[20, l]|}
				
				\hline \multicolumn{3}{|c|}{\textbf{fights}}	 \\[3pt] \hline
				\endfirsthead
				
				\hline \multicolumn{3}{|c|}{\textbf{fights}}	 \\[3pt] \hline
				\endhead
				
				\hline 
				\endlastfoot
				
				\cellcolor{LightGreen}fight\textunderscore id & INT8	&  	jedinstveni identifikator borbe 	\\ \hline
				start\textunderscore time	& TIMESTAMP &   trenutak početka borbe	\\ \hline 
				\cellcolor{LightBlue}winner\textunderscore user\textunderscore id & INT8 & jedinstveni identifikator korisnika koji je pobijedio u borbi (users.user\textunderscore id) \\ \hline
				
				
			\end{longtabu}
		
			\textnormal{\textbf{User\textunderscore card\textunderscore fight} \quad Ova tablica nastaje zbog \textit{Many-to-Many-to-Many} veze između entiteta Users, Fight i Location\textunderscore cards. Sadrži njihove primarne ključeve te je time i povezana s njima, a prikazuje borbu u kojoj je korisnik sudjelovao s određenom karticom.}
		
			\begin{longtabu} to \textwidth {|X[6, l]|X[6, l]|X[20, l]|}
				
				\hline \multicolumn{3}{|c|}{\textbf{user\textunderscore card\textunderscore fight}}	 \\[3pt] \hline
				\endfirsthead
				
				\hline \multicolumn{3}{|c|}{\textbf{user\textunderscore card\textunderscore fight}}	 \\[3pt] \hline
				\endhead
				
				\hline 
				\endlastfoot
				
				\cellcolor{LightGreen}fight\textunderscore id & INT8	&  	jedinstveni identifikator borbe (fights.fight\textunderscore id) 	\\ \hline
				
				\cellcolor{LightGreen}user\textunderscore id & INT8	&  	jedinstveni identifikator igrača (users.user\textunderscore id) 	\\ \hline
				
				\cellcolor{LightGreen}lcd\textunderscore id & INT8	&  	jedinstveni identifikator kartice lokacije (location\textunderscore cards.lcd\textunderscore id) 	\\ \hline
				
				
			\end{longtabu}
			
			\subsection{Dijagram baze podataka}
				\textit{ U ovom potpoglavlju potrebno je umetnuti dijagram baze podataka. Primarni i strani ključevi moraju biti označeni, a tablice povezane. Bazu podataka je potrebno normalizirati. Podsjetite se kolegija "Baze podataka".}
				
				\begin{figure}[H]
					\centering
					\includegraphics[scale=0.4]{slike/ERModel} 
					\caption{ER dijagram baze podataka}
					\label{fig:ER}
				\end{figure}
			
			\eject
			
			
		\section{Dijagram razreda}
		
			\textit{Potrebno je priložiti dijagram razreda s pripadajućim opisom. Zbog preglednosti je moguće dijagram razlomiti na više njih, ali moraju biti grupirani prema sličnim razinama apstrakcije i srodnim funkcionalnostima.}\\
			
			\textbf{\textit{dio 1. revizije}}\\
			
			\textit{Prilikom prve predaje projekta, potrebno je priložiti potpuno razrađen dijagram razreda vezan uz \textbf{generičku funkcionalnost} sustava. Ostale funkcionalnosti trebaju biti idejno razrađene u dijagramu sa sljedećim komponentama: nazivi razreda, nazivi metoda i vrste pristupa metodama (npr. javni, zaštićeni), nazivi atributa razreda, veze i odnosi između razreda.}\\
			
			\textbf{\textit{dio 2. revizije}}\\			
			
			\textit{Prilikom druge predaje projekta dijagram razreda i opisi moraju odgovarati stvarnom stanju implementacije}
			
			
			
			\eject
		
		\section{Dijagram stanja}
			
			
			\textbf{\textit{dio 2. revizije}}\\
			
			\textit{Potrebno je priložiti dijagram stanja i opisati ga. Dovoljan je jedan dijagram stanja koji prikazuje \textbf{značajan dio funkcionalnosti} sustava. Na primjer, stanja korisničkog sučelja i tijek korištenja neke ključne funkcionalnosti jesu značajan dio sustava, a registracija i prijava nisu. }
			
			
			\eject 
		
		\section{Dijagram aktivnosti}
			
			\textbf{\textit{dio 2. revizije}}\\
			
			 \textit{Potrebno je priložiti dijagram aktivnosti s pripadajućim opisom. Dijagram aktivnosti treba prikazivati značajan dio sustava.}
			
			\eject
		\section{Dijagram komponenti}
		
			\textbf{\textit{dio 2. revizije}}\\
		
			 \textit{Potrebno je priložiti dijagram komponenti s pripadajućim opisom. Dijagram komponenti treba prikazivati strukturu cijele aplikacije.}