\chapter{Specifikacija programske potpore}
		
	\section{Funkcionalni zahtjevi}
			
			\textbf{\textit{dio 1. revizije}}\\
			
			\textit{Navesti \textbf{dionike} koji imaju \textbf{interes u ovom sustavu} ili  \textbf{su nositelji odgovornosti}. To su prije svega korisnici, ali i administratori sustava, naručitelji, razvojni tim.}\\
				
			\textit{Navesti \textbf{aktore} koji izravno \textbf{koriste} ili \textbf{komuniciraju sa sustavom}. Oni mogu imati inicijatorsku ulogu, tj. započinju određene procese u sustavu ili samo sudioničku ulogu, tj. obavljaju određeni posao. Za svakog aktora navesti funkcionalne zahtjeve koji se na njega odnose.}\\
			
			
			\noindent \textbf{Dionici:}
			
			\begin{packed_enum}
				
				\item Dionik 1
				\item Dionik 2				
				\item ...
				
			\end{packed_enum}
			
			\noindent \textbf{Aktori i njihovi funkcionalni zahtjevi:}
			
			
			\begin{packed_enum}
				\item  \underbar{Aktor 1 (inicijator) može:}
				
				\begin{packed_enum}
					
					\item funkcionalnost 1
					\item funkcionalnost 2
					\begin{packed_enum}
						
						\item  podfunkcionalnost 1 
						\item  podfunkcionalnost 2
				
					\end{packed_enum}
					\item  funkcionalnost 3
					
				\end{packed_enum}
			
				\item  \underbar{Aktor 2 (sudionik) može:}
				
				\begin{packed_enum}
					
					\item funkcionalnost 1
					\item funkcionalnost 2
					
				\end{packed_enum}
			\end{packed_enum}
			
			\eject 
			
			
				
			\subsection{Obrasci uporabe}
				
				\textbf{\textit{dio 1. revizije}}
				
				\subsubsection{Opis obrazaca uporabe}
					\textit{Funkcionalne zahtjeve razraditi u obliku obrazaca uporabe. Svaki obrazac je potrebno razraditi prema donjem predlošku. Ukoliko u nekom koraku može doći do odstupanja, potrebno je to odstupanje opisati i po mogućnosti ponuditi rješenje kojim bi se tijek obrasca vratio na osnovni tijek.}\\
					
				
				\noindent \underbar{\textbf{UC11 - Pregled popisa korisnika}}
				\begin{packed_item}
					
					\item \textbf{Glavni sudionik: }Administrator
					\item  \textbf{Cilj:} Pregled popisa korisnika
					\item  \textbf{Sudionici:} Baza podataka
					\item  \textbf{Preduvjet:} Korisnik je prijavljen kao administrator
					\item  \textbf{Opis osnovnog tijeka:}
					
					\item[] \begin{packed_enum}
						
						\item Administrator klikom na opciju za pregled registriranih korisnika šalje upit bazi podataka.
						\item Baza podataka vraća popis registriranih korisnika.
						\item Popis korisnika se administratoru prikazuje u web sučelju.
					\end{packed_enum}
					
				\end{packed_item}
			
				\noindent \underbar{\textbf{UC11.1 - Brisanje korisnika s popisa}}
				\begin{packed_item}
					
					\item \textbf{Glavni sudionik: }Administrator
					\item  \textbf{Cilj:} Izbrisati korisnika s popisa iz baze podataka
					\item  \textbf{Sudionici:} Baza podataka
					\item  \textbf{Preduvjet:} Korisnik je prijavljen kao administrator
					\item  \textbf{Opis osnovnog tijeka:}
					
					\item[] \begin{packed_enum}
						
						\item Administrator odabire opciju za pregled registriranih korisnika.
						\item Popis korisnika se administratoru prikazuje u web sučelju.
						\item Pored imena korisnika administrator odabire opciju za brisanje korisnika.
						\item Korisnik se briše iz baze podataka.
						\item Ažurira se popis korisnika u web sučelju.
					\end{packed_enum}
					
				\end{packed_item}
			
				\noindent \underbar{\textbf{UC11.2 - Uređivanje osobnih podataka korisnika}}
				\begin{packed_item}
					
					\item \textbf{Glavni sudionik: }Administrator
					\item  \textbf{Cilj:} Urediti osobne podatke korisnika
					\item  \textbf{Sudionici:} Baza podataka
					\item  \textbf{Preduvjet:} Korisnik je prijavljen kao administrator
					\item  \textbf{Opis osnovnog tijeka:}
					
					\item[] \begin{packed_enum}
						
						\item Administrator odabire opciju za pregled registriranih korisnika.
						\item Popis korisnika se administratoru prikazuje u web sučelju.
						\item Administrator klikom na ime korisnika šalje upit bazi podataka.
						\item Podaci o korisniku se prikazuju administratoru.
						\item Administrator mijenja korisničke podatke i odabire opciju za spremanje.
						\item Aplikacija šalje nove korisničke podatke bazi podataka.
					\end{packed_enum}
					
					\item  \textbf{Opis mogućih odstupanja:}
					
					\item[] \begin{packed_item}
						
						\item[3.a] Drugi administrator je u međuvremenu izbrisao korisnika
						\item[] \begin{packed_enum}
							
							\item Administratoru se ispisuje poruka da taj korisnik više ne postoji.
							
						\end{packed_enum}
						\item[6.a] Drugi administrator je u međuvremenu izbrisao korisnika
						\item[] \begin{packed_enum}
							
							\item Administratoru se ispisuje poruka da taj korisnik više ne postoji.
							
						\end{packed_enum}
						
					\end{packed_item}
				\end{packed_item}
				
				\noindent \underbar{\textbf{UC11.3 - Isključivanje igrača iz igre}}
				\begin{packed_item}
					
					\item \textbf{Glavni sudionik: }Administrator
					\item  \textbf{Cilj:} Isključiti igrača iz igre trajno ili privremeno
					\item  \textbf{Sudionici:} Baza podataka
					\item  \textbf{Preduvjet:} Korisnik je prijavljen kao administrator
					\item  \textbf{Opis osnovnog tijeka:}
					
					\item[] \begin{packed_enum}
						
						\item Administrator odabire opciju za pregled registriranih korisnika.
						\item Popis korisnika se administratoru prikazuje u web sučelju.
						\item Pored imena korisnika administrator odabire opciju za isključenje korisnika.
						\item Administratoru se prikazuje obrazac u kojem treba odabrati opciju za trajno ili privremeno isključenje uz navedeno vrijeme.
						\item Administrator potvrđuje odabir.
						\item Aplikacija zabilježi isključenog korisnika u bazi podataka.
					\end{packed_enum}
					
					\item  \textbf{Opis mogućih odstupanja:}
					
					\item[] \begin{packed_item}
						
						\item[3.a] Drugi administrator je u međuvremenu izbrisao korisnika
						\item[] \begin{packed_enum}
							
							\item Administratoru se ispisuje poruka da taj korisnik više ne postoji.
							
						\end{packed_enum}
						\item[5.a] Drugi administrator je u međuvremenu izbrisao korisnika
						\item[] \begin{packed_enum}
							
							\item Podaci o isključenom korisniku se ne bilježe u bazi podataka.
							\item Administratoru se ispisuje poruka da taj korisnik više ne postoji.
							
						\end{packed_enum}
						
					\end{packed_item}
				\end{packed_item}
				
				\noindent \underbar{\textbf{UC12 - Pregled popisa lokacija}}
				\begin{packed_item}
					
					\item \textbf{Glavni sudionik: }Administrator
					\item  \textbf{Cilj:} Pregled popisa lokacija igračkih karata
					\item  \textbf{Sudionici:} Baza podataka
					\item  \textbf{Preduvjet:} Korisnik je prijavljen kao administrator
					\item  \textbf{Opis osnovnog tijeka:}
					
					\item[] \begin{packed_enum}
						
						\item Administrator klikom na opciju za pregled lokacija šalje upit bazi podataka.
						\item Baza podataka vraća popis lokacija na kojima su karte.
						\item Popis lokacija se administratoru prikazuje u web sučelju.
					\end{packed_enum}
				\end{packed_item}
				
				\noindent \underbar{\textbf{UC12.1 - Uređivanje podataka lokacije}}
				\begin{packed_item}
					
					\item \textbf{Glavni sudionik: }Administrator
					\item  \textbf{Cilj:} Urediti podatke o lokaciji karte
					\item  \textbf{Sudionici:} Baza podataka
					\item  \textbf{Preduvjet:} Korisnik je prijavljen kao administrator
					\item  \textbf{Opis osnovnog tijeka:}
					
					\item[] \begin{packed_enum}
						
						\item Administrator odabire opciju za pregled lokacija.
						\item Popis lokacija se administratoru prikazuje u web sučelju.
						\item Administrator klikom na ime lokacije šalje upit bazi podataka.
						\item Podaci o lokaciji se prikazuju administratoru.
						\item Administrator mijenja podatke o lokaciji i odabire opciju za spremanje.
						\item Aplikacija šalje nove podatke o lokaciji bazi podataka.
					\end{packed_enum}
					
					\item  \textbf{Opis mogućih odstupanja:}
					
					\item[] \begin{packed_item}
						
						\item[3.a] Drugi administrator je u međuvremenu izbrisao lokaciju
						\item[] \begin{packed_enum}
							
							\item Administratoru se ispisuje poruka da ta lokacija više ne postoji.
							
						\end{packed_enum}
						\item[6.a] Drugi administrator je u međuvremenu izbrisao lokaciju
						\item[] \begin{packed_enum}
							
							\item Administratoru se ispisuje poruka da ta lokacija više ne postoji.
							
						\end{packed_enum}
						
					\end{packed_item}
				\end{packed_item}
				
				\noindent \underbar{\textbf{UC12.2 - Brisanje lokacije s popisa}}
				\begin{packed_item}
					
					\item \textbf{Glavni sudionik: }Administrator
					\item  \textbf{Cilj:} Izbrisati lokaciju s popisa iz baze podataka
					\item  \textbf{Sudionici:} Baza podataka
					\item  \textbf{Preduvjet:} Korisnik je prijavljen kao administrator
					\item  \textbf{Opis osnovnog tijeka:}
					
					\item[] \begin{packed_enum}
						
						\item Administrator odabire opciju za pregled lokacija.
						\item Popis lokacija se administratoru prikazuje u web sučelju.
						\item Pored imena lokacije administrator odabire opciju za brisanje lokacije.
						\item Lokacija se briše iz baze podataka.
						\item Ažurira se popis lokacija u web sučelju.
					\end{packed_enum}
				\end{packed_item}
				
				\noindent \underbar{\textbf{UC13 - Pregled prijava kartografa}}
				\begin{packed_item}
					
					\item \textbf{Glavni sudionik: }Administrator
					\item  \textbf{Cilj:} Pregledati popis prijava za ulogu kartografa
					\item  \textbf{Sudionici:} Baza podataka
					\item  \textbf{Preduvjet:} Korisnik je prijavljen kao administrator
					\item  \textbf{Opis osnovnog tijeka:}
					
					\item[] \begin{packed_enum}
						
						\item Administrator klikom na opciju za pregled prijava kartografa šalje upit bazi podataka.
						\item Baza podataka vraća popis prijava kartografa.
						\item Popis prijava se administratoru prikazuje u web sučelju.
					\end{packed_enum}
					
					\item  \textbf{Opis mogućih odstupanja:}
					
					\item[] \begin{packed_item}
						
						\item[2.a] Nema prijava kartografa
						\item[] \begin{packed_enum}
							
							\item Administratoru se ispisuje poruka da nema prijava kartografa

						\end{packed_enum}
						
					\end{packed_item}
				\end{packed_item}
				
				\noindent \underbar{\textbf{UC13.1 - Prihvat prijave kartografa}}
				\begin{packed_item}
					
					\item \textbf{Glavni sudionik: }Administrator
					\item  \textbf{Cilj:} Prihvatiti prijavu za ulogu kartografa
					\item  \textbf{Sudionici:} Baza podataka
					\item  \textbf{Preduvjet:} Korisnik je prijavljen kao administrator
					\item  \textbf{Opis osnovnog tijeka:}
					
					\item[] \begin{packed_enum}
						
						\item Administrator klikom na opciju za pregled prijava kartografa šalje upit bazi podataka.
						\item Popis prijava se administratoru prikazuje u web sučelju.
						\item Administrator prihvaća prijavu klikom na opciju za prihvat pored prijave kartografa.
						\item Novi kartograf se pohranjuje u bazu podataka.
						\item Ažurira se popis prijava kartografa u web sučelju.
					\end{packed_enum}
					
					\item  \textbf{Opis mogućih odstupanja:}
					
					\item[] \begin{packed_item}
						
						\item[3.a] Drugi administrator je u međuvremenu prihvatio ili odbio prijavu
						\item[] \begin{packed_enum}
							
							\item Administratoru se ispisuje poruka da prijava nije više valjana
							
						\end{packed_enum}
						
					\end{packed_item}
				\end{packed_item}
				
				\noindent \underbar{\textbf{UC13.2 - Odbijanje prijave kartografa}}
				\begin{packed_item}
					
					\item \textbf{Glavni sudionik: }Administrator
					\item  \textbf{Cilj:} Odbiti prijavu za ulogu kartografa
					\item  \textbf{Sudionici:} Baza podataka
					\item  \textbf{Preduvjet:} Korisnik je prijavljen kao administrator
					\item  \textbf{Opis osnovnog tijeka:}
					
					\item[] \begin{packed_enum}
						
						\item Administrator klikom na opciju za pregled prijava kartografa šalje upit bazi podataka.
						\item Popis prijava se administratoru prikazuje u web sučelju.
						\item Administrator odbija prijavu klikom na opciju za odbijanje pored prijave kartografa.
						\item Prijava kartografa se briše iz baze podataka.
						\item Ažurira se popis prijava kartografa u web sučelju.
					\end{packed_enum}
					
					\item  \textbf{Opis mogućih odstupanja:}
					
					\item[] \begin{packed_item}
						
						\item[3.a] Drugi administrator je u međuvremenu prihvatio ili odbio prijavu
						\item[] \begin{packed_enum}
							
							\item Administratoru se ispisuje poruka da prijava nije više valjana.
							
						\end{packed_enum}
						
					\end{packed_item}
				\end{packed_item}
				
				\noindent \underbar{\textbf{UC14 - Prikaz prijave na karti}}
				\begin{packed_item}
					
					\item \textbf{Glavni sudionik: }Kartograf
					\item  \textbf{Cilj:} Prikazati novu prijavu lokacije na karti
					\item  \textbf{Sudionici:} Baza podataka
					\item  \textbf{Preduvjet:} Korisnik je prijavljen kao kartograf
					\item  \textbf{Opis osnovnog tijeka:}
					
					\item[] \begin{packed_enum}
						
						\item Kartograf klikom na opciju za pregled prijava lokacija šalje upit bazi podataka.
						\item Prijavljene lokacije se kartografu prikažu na karti.
					\end{packed_enum}
				\end{packed_item}
				
				\noindent \underbar{\textbf{UC14.1 - Prihvat prijave lokacije}}
				\begin{packed_item}
					
					\item \textbf{Glavni sudionik: }Kartograf
					\item  \textbf{Cilj:} Prihvatiti prijavu lokacije
					\item  \textbf{Sudionici:} Baza podataka
					\item  \textbf{Preduvjet:} Korisnik je prijavljen kao kartograf
					\item  \textbf{Opis osnovnog tijeka:}
					
					\item[] \begin{packed_enum}
						
						\item Kartograf klikom na opciju za pregled prijava lokacija šalje upit bazi podataka.
						\item Prijavljene lokacije se kartografu prikažu na karti.
						\item Kartograf klikom na lokaciju otvara prozor s opisom lokacije.
						\item Kartograf prihvaća lokaciju klikom na opciju za prihvat.
						\item Prijava lokacije se briše iz baze podataka.
						\item Nova lokacija se dodaje u bazu podataka
						\item Ažurira se popis prijava lokacija u web sučelju.
					\end{packed_enum}
					
					\item  \textbf{Opis mogućih odstupanja:}
					
					\item[] \begin{packed_item}
						
						\item[3.a] Drugi kartograf je u međuvremenu prihvatio ili odbio lokaciju
						\item[] \begin{packed_enum}
							
							\item Administratoru se ispisuje poruka da prijava nije više valjana.
							
						\end{packed_enum}
						
					\end{packed_item}
				\end{packed_item}
				
				\noindent \underbar{\textbf{UC14.2 - Odbijanje prijave lokacije}}
				\begin{packed_item}
					
					\item \textbf{Glavni sudionik: }Kartograf
					\item  \textbf{Cilj:} Odbiti prijavu lokacije
					\item  \textbf{Sudionici:} Baza podataka
					\item  \textbf{Preduvjet:} Korisnik je prijavljen kao kartograf
					\item  \textbf{Opis osnovnog tijeka:}
					
					\item[] \begin{packed_enum}
						
						\item Kartograf klikom na opciju za pregled prijava lokacija šalje upit bazi podataka.
						\item Prijavljene lokacije se kartografu prikažu na karti.
						\item Kartograf klikom na lokaciju otvara prozor s opisom lokacije.
						\item Kartograf odbija prijavu lokacije klikom na opciju za odbijanje.
						\item Lokacija se briše iz popisa prijava lokacija u bazi podataka i sa karte u web sučelju.
					\end{packed_enum}
					
					\item  \textbf{Opis mogućih odstupanja:}
					
					\item[] \begin{packed_item}
						
						\item[3.a] Drugi kartograf je u međuvremenu prihvatio ili odbio lokaciju
						\item[] \begin{packed_enum}
							
							\item Kartografu se ispisuje poruka da prijava nije više valjana.
							
						\end{packed_enum}
						
					\end{packed_item}
				\end{packed_item}
				
				\noindent \underbar{\textbf{UC14.3 - Uređivanje prijave lokacije}}
				\begin{packed_item}
					
					\item \textbf{Glavni sudionik: }Kartograf
					\item  \textbf{Cilj:} Urediti prijavu lokacije
					\item  \textbf{Sudionici:} Baza podataka
					\item  \textbf{Preduvjet:} Korisnik je prijavljen kao kartograf
					\item  \textbf{Opis osnovnog tijeka:}
					
					\item[] \begin{packed_enum}
						
						\item Kartograf klikom na opciju za pregled prijava lokacija šalje upit bazi podataka.
						\item Prijavljene lokacije se kartografu prikažu na karti.
						\item Kartograf klikom na lokaciju otvara prozor s opisom lokacije.
						\item Kartograf uređuje podatke o prijavi lokacije i odabire opciju za spremanje.
						\item Prijava lokacije se ažurira u bazi podataka.
						\item Aplikacija ažurira podatke o prijavi lokacije.
					\end{packed_enum}
					
					\item  \textbf{Opis mogućih odstupanja:}
					
					\item[] \begin{packed_item}
						
						\item[4.a] Drugi kartograf je u međuvremenu potvrdio ili odbio prijavu lokacije
						\item[] \begin{packed_enum}
							
							\item Kartografu se ispisuje poruka da prijava nije više valjana.
							
						\end{packed_enum}
						
					\end{packed_item}
				\end{packed_item}
				
				\noindent \underbar{\textbf{UC14.4 - Potvrda prijave s terena}}
				\begin{packed_item}
					
					\item \textbf{Glavni sudionik: }Kartograf
					\item  \textbf{Cilj:} Potvrditi prijavu lokacije posjetom lokaciji
					\item  \textbf{Sudionici:} Baza podataka
					\item  \textbf{Preduvjet:} Korisnik je prijavljen kao kartograf
					\item  \textbf{Opis osnovnog tijeka:}
					
					\item[] \begin{packed_enum}
						
						\item Kartograf klikom na opciju za pregled prijava lokacija šalje upit bazi podataka.
						\item Prijavljene lokacije se kartografu prikažu na karti.
						\item Kartograf klikom na lokaciju otvara prozor s opisom lokacije.
						\item Kartograf označava lokaciju da je potrebna provjera s terena.
						\item Prijava lokacije se ažurira u bazi podataka.
						\item Aplikacija ažurira podatke o prijavi lokacije.
					\end{packed_enum}
					
					\item  \textbf{Opis mogućih odstupanja:}
					
					\item[] \begin{packed_item}
						
						\item[5.a] Drugi kartograf je u međuvremenu potvrdio ili odbio prijavu lokacije
						\item[] \begin{packed_enum}
							
							\item Kartografu se ispisuje poruka da prijava nije više valjana.
							
						\end{packed_enum}
						
					\end{packed_item}
				\end{packed_item}
				
				\noindent \underbar{\textbf{UC15 - Pregled mape s najkraćim putom do prijavljenih lokacija}}
				\begin{packed_item}
					
					\item \textbf{Glavni sudionik: }Kartograf
					\item  \textbf{Cilj:} Prikazati na karti najkraći put do prijavljene lokacije
					\item  \textbf{Sudionici:} Baza podataka, OSMR sustav
					\item  \textbf{Preduvjet:} Korisnik je prijavljen kao kartograf
					\item  \textbf{Opis osnovnog tijeka:}
					
					\item[] \begin{packed_enum}
						
						\item Kartograf na karti odabire opciju za prikaz najkraćeg puta
						\item Aplikacija pomoću sustava OSRM riješi problem trgovačkog putnika za sve lokacije označene za potvrdu prijave s terena
						\item Kartografu se na karti pokaže rješenje trgovačkog putnika
					\end{packed_enum}
					
					\item  \textbf{Opis mogućih odstupanja:}
					
					\item[] \begin{packed_item}
						
						\item[2.a] OSRM servis je nedostupan
						\item[] \begin{packed_enum}
							
							\item Kartografu se ispisuje poruka da se ne može izračunati najkraći put
							
						\end{packed_enum}
						
					\end{packed_item}
				\end{packed_item}
					
				\subsubsection{Dijagrami obrazaca uporabe}
					
					\textit{Prikazati odnos aktora i obrazaca uporabe odgovarajućim UML dijagramom. Nije nužno nacrtati sve na jednom dijagramu. Modelirati po razinama apstrakcije i skupovima srodnih funkcionalnosti.}
				\eject		
				
			\subsection{Sekvencijski dijagrami}
				
				\textbf{\textit{dio 1. revizije}}\\
				
				\textit{Nacrtati sekvencijske dijagrame koji modeliraju najvažnije dijelove sustava (max. 4 dijagrama). Ukoliko postoji nedoumica oko odabira, razjasniti s asistentom. Uz svaki dijagram napisati detaljni opis dijagrama.}
				\eject
	
		\section{Ostali zahtjevi}
		
			\textbf{\textit{dio 1. revizije}}\\
		 
			 \textit{Nefunkcionalni zahtjevi i zahtjevi domene primjene dopunjuju funkcionalne zahtjeve. Oni opisuju \textbf{kako se sustav treba ponašati} i koja \textbf{ograničenja} treba poštivati (performanse, korisničko iskustvo, pouzdanost, standardi kvalitete, sigurnost...). Primjeri takvih zahtjeva u Vašem projektu mogu biti: podržani jezici korisničkog sučelja, vrijeme odziva, najveći mogući podržani broj korisnika, podržane web/mobilne platforme, razina zaštite (protokoli komunikacije, kriptiranje...)... Svaki takav zahtjev potrebno je navesti u jednoj ili dvije rečenice.}
			 
			 
			 
	