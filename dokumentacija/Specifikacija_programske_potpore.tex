\chapter{Specifikacija programske potpore}
		
	\section{Funkcionalni zahtjevi}
			
			\textbf{\textit{dio 1. revizije}}\\
			
			\textit{Navesti \textbf{dionike} koji imaju \textbf{interes u ovom sustavu} ili  \textbf{su nositelji odgovornosti}. To su prije svega korisnici, ali i administratori sustava, naručitelji, razvojni tim.}\\
				
			\textit{Navesti \textbf{aktore} koji izravno \textbf{koriste} ili \textbf{komuniciraju sa sustavom}. Oni mogu imati inicijatorsku ulogu, tj. započinju određene procese u sustavu ili samo sudioničku ulogu, tj. obavljaju određeni posao. Za svakog aktora navesti funkcionalne zahtjeve koji se na njega odnose.}\\
			
			
			\noindent \textbf{Dionici:}
			
			\begin{packed_enum}
				
				\item Razvojni tim
				\item Igrači				
				\item Kartografi
				\item Admin
				
			\end{packed_enum}
			
			\noindent \textbf{Aktori i njihovi funkcionalni zahtjevi:}
			
			
			\begin{packed_enum}
				\item  \underbar{Neregistrirani korisnik (inicijator) može:}
				
				\begin{packed_enum}
					
					\item se registrirati u sustav kao:
					\begin{packed_enum}
						
						\item  igrač (stvaranje korisničkog računa, potrebni sljedeći podaci: korisničko ime, lozinka, e-mail adresa, fotografija)
						\item  kartograf (stvaranje korisničkog računa, potrebni sljedeći podaci: IBAN računa za uplatu, fotografija osobne iskaznice)
				
					\end{packed_enum}
					
				\end{packed_enum}
			
				\item  \underbar{Igrač (inicijator) može:}
				
				\begin{packed_enum}
					
					\item se prijaviti u sustav koristeći email i lozinku
					\item se odjaviti iz sustava
					\item pregledati vlastiti profil
					\begin{packed_enum}
						\item pristupiti prikazu svih sakupljenih karti
						\item pristupiti prikazu statistike 10 zadnjih odigranih borbi
					\end{packed_enum}
				\item pregledati profil drugog igrača
				\begin{packed_enum}
					\item pristupiti prikazu svih sakupljenih karti
					\item pristupiti prikazu statistike 10 zadnjih odigranih borbi
				\end{packed_enum}
			\item pregledati detalje pojedinačne karte iz kolekcije
			\item pregledati popis igrača u blizini
			\item pristupiti borbi kroz nasumičnu tražilicu
			\item se boriti u kartaškom dvoboju protiv drugog igrača
			\begin{packed_enum}
				\item odabrati karte za borbu iz svoje kolekcije sakupljenih karata
				\item odigrati kartaški dvoboj
			\end{packed_enum}
		\item prijaviti novu lokaciju
		\item dobiti novu kartu ako ju je igrač prijavio te je ona odobrena bez izmjena
		\item pristupiti pregledu rang ljestvice svih igrača
					
				\end{packed_enum}
			\item  \underbar{Kartograf (inicijator) može:}
			
			\begin{packed_enum}
				\item se prijaviti u sustav preko broja osobne iskaznice
				\item se odjaviti iz sustava
				\item pristupiti prikazu svih prijava:
				\begin{packed_enum}
					\item prihvatiti lokaciju iz prijave
					\item odbiti lokaciju iz prijave
					\item može izmijeniti prijavu te zatim lokaciju prihvatiti
					\item može odabrati opciju da je potrebna potvrda s terena, te ga sustav na mapi vodi najkraćim putom od vlastite lokacije do prijavljene lokacije
				\end{packed_enum}
			\end{packed_enum}
		\item  \underbar{Administrator (inicijator) uz mogućnosti kartografa i igrača može:}
		\begin{packed_enum}
			\item pristupiti pregledu popisa svih korisnika
			\begin{packed_enum}
				\item brisati korisnike s popisa, odnosno ukloniti korisnike iz igre
				\item uređivati osobne podatke svih korisnika
				\item isključiti igrače privremeno iz igre
			\end{packed_enum}
		\item pristupiti pregledu svih lokacija (karti) u igri
		\begin{packed_enum}
			\item uređivati postojeće lokacije (karte)
			\item brisati postojeće lokacije (karte)
		\end{packed_enum}
	\item pristupiti pregledu prijava kartografa
	\begin{packed_enum}
		\item prihvatiti prijavu
		 \item odbiti prijavu
	\end{packed_enum}
		\end{packed_enum}
	\item  \underbar{Baza podataka (sudionik):}
	\begin{packed_enum}
		\item pohranjuje sve podatke o svim korisnicima, njihovim kolekcijama i njihovim ovlastima
		\item pohranjuje sve podatke o postojećim lokacijama i prijavama lokacija
	\end{packed_enum}
\item  \underbar{OSRM (sudionik) :}
		\begin{packed_enum}
			\item omogućava pronalazak najkraćeg puta dvije lokacije na mapi
		\end{packed_enum}
			\end{packed_enum}
			
			\eject 
			
			
				
			\subsection{Obrasci uporabe}
				
				\textbf{\textit{dio 1. revizije}}
				
				\subsubsection{Opis obrazaca uporabe}
					\textit{Funkcionalne zahtjeve razraditi u obliku obrazaca uporabe. Svaki obrazac je potrebno razraditi prema donjem predlošku. Ukoliko u nekom koraku može doći do odstupanja, potrebno je to odstupanje opisati i po mogućnosti ponuditi rješenje kojim bi se tijek obrasca vratio na osnovni tijek.}\\
					

					\noindent \underbar{\textbf{UC1 -Registracija}}
					\begin{packed_item}
	
						\item \textbf{Glavni sudionik: }Neregistrirani korisnik
						\item  \textbf{Cilj:} Stvoriti korisnički račun za pristup sustavu
						\item  \textbf{Sudionici:} Baza podataka
						\item  \textbf{Preduvjet:} -
						\item  \textbf{Opis osnovnog tijeka:}
						
						\item[] \begin{packed_enum}
	
							\item Korisnik odabire opciju za registraciju.
							\item Korisnik unosi potrebne korisničke podatke.
							\item Korisnik prima e-mail za potvrdu registracije.
							\item Potvrdom e-maila korisnik prima obavijest o uspješnoj registraciji.
						\end{packed_enum}
						
						\item  \textbf{Opis mogućih odstupanja:}
						
						\item[] \begin{packed_item}
	
							\item[2.a] Odabir već zauzetog korisničkog imena i/ili e-maila, unos korisničkog podatka u nedozvoljenom formatu ili neispravnoga e-maila.
							\item[] \begin{packed_enum}
								
								\item Sustav obavještava korisnika o neuspjelom upisu i vraća ga na stranicu za registraciju.
								\item Korisnik mijenja potrebne podatke te završava unos ili odustaje od registracije.
								
							\end{packed_enum}
							
						\end{packed_item}
					\end{packed_item}
				\noindent \underbar{\textbf{UC1.2 -Registracija kartografa}}
				\begin{packed_item}
					
					\item \textbf{Glavni sudionik: }Neregistrirani korisnik
					\item  \textbf{Cilj:} Stvoriti korisnički račun za pristup sustavu kao kartograf
					\item  \textbf{Sudionici:} Baza podataka
					\item  \textbf{Preduvjet:} -
					\item  \textbf{Opis osnovnog tijeka:}
					
					\item[] \begin{packed_enum}
						
						\item Korisnik odabire opciju za registraciju.
						\item Korisnik unosi potrebne korisničke podatke.
					\end{packed_enum}
					
					\item  \textbf{Opis mogućih odstupanja:}
					
					\item[] \begin{packed_item}
						
						\item[2.a] Odbijenica od strane administratora
						\item[] \begin{packed_enum}
							
							\item Prihvatiti odbijenicu i pokusati se prijaviti drugi put
							
						\end{packed_enum}
						
					\end{packed_item}
				\end{packed_item}
			\noindent \underbar{\textbf{UC2 -Prijava}}
			\begin{packed_item}
				
				\item \textbf{Glavni sudionik: }Registrirani korisnik
				\item  \textbf{Cilj:} Dobiti pristup korisničkom sučelju
				\item  \textbf{Sudionici:} Baza podataka
				\item  \textbf{Preduvjet:} Registracija
				\item  \textbf{Opis osnovnog tijeka:}
				
				\item[] \begin{packed_enum}
					
					\item Unos emaila i lozinke.
					\item Potvrda o ispravnosti unesenih podataka.
					\item Pristup korisničkim funkcijama.
				\end{packed_enum}
				
				\item  \textbf{Opis mogućih odstupanja:}
				
				\item[] \begin{packed_item}
					
					\item[2.a] Neispravan email ili lozinka
					\item[] \begin{packed_enum}
						
						\item  Sustav obavještava korisnika o neuspješnoj prijavi. Korisnik pokušava ponovno
						\item ustav obavještava korisnika o neuspješnoj prijavi. Korisnik odustaje.
						
					\end{packed_enum}
					
				\end{packed_item}
			\end{packed_item}
		\noindent \underbar{\textbf{UC3 -Odjava iz sustava}}
		\begin{packed_item}
			
			\item \textbf{Glavni sudionik: }Registrirani korisnik
			\item  \textbf{Cilj:} Završetak korištenja korisničkog sučelja.
			\item  \textbf{Sudionici:} Baza podataka
			\item  \textbf{Preduvjet:} Prijava
			\item  \textbf{Opis osnovnog tijeka:}
			
			\item[] \begin{packed_enum}
				
				\item Korisnik odabire opciju za odjavu.
				\item Sustav odjavljuje korisnika i vraća ga na početnu stranicu.
			\end{packed_enum}
			
		\end{packed_item}
	
	\noindent \underbar{\textbf{UC4 - Pregled vlastitog profila}}
	\begin{packed_item}
		
		\item \textbf{Glavni sudionik: }Registrirani korisnik
		\item  \textbf{Cilj:} Pregled vlastitog profila
		\item  \textbf{Sudionici:} Baza podataka
		\item  \textbf{Preduvjet:} Prijava
		\item  \textbf{Opis osnovnog tijeka:}
		
		\item[] \begin{packed_enum}
			
			\item Korisnik odabire opciju za pregled vlastitog profila.
			\item Aplikacija prikazuje osobne podatke, vlastiti rang i statistiku zadnjih 10 borbi.
		\end{packed_enum}
	\end{packed_item}

\noindent \underbar{\textbf{UC4.1 -Pregled svih sakupljenih karata}}
\begin{packed_item}
	
	\item \textbf{Glavni sudionik: }Registrirani korisnik
	\item  \textbf{Cilj:} Pregledati sve sakupljene karte
	\item  \textbf{Sudionici:} Baza podataka
	\item  \textbf{Preduvjet:} Pregled vlastitog ili tuđeg profila
	\item  \textbf{Opis osnovnog tijeka:}
	
	\item[] \begin{packed_enum}
		
		\item Korisnik odlazi na svoj ili tuđi profil
		\item Odabire opciju pregleda kolekcije karata
		\item Aplikacija prikazuje sve karte koje taj profil ima u svojoj kolekciji
	\end{packed_enum}
\end{packed_item}

\noindent \underbar{\textbf{UC5 -Pregled profila drugog igrača}}
\begin{packed_item}
	
	\item \textbf{Glavni sudionik: }Registrirani korisnik
	\item  \textbf{Cilj:} Pregled podataka drugog igrača
	\item  \textbf{Sudionici:} Baza podataka
	\item  \textbf{Preduvjet:} Pregled igrača u blizini ili pregled rang ljestvice
	\item  \textbf{Opis osnovnog tijeka:}
	
	\item[] \begin{packed_enum}
		
		\item Korisnik odabire igrača s popisa čiji profil želi otvoriti
		\item Prikazuju se detalji drugog igrača: ime, rang, statistika zadnjih 10 borbi, te gumb za kolekciju
	\end{packed_enum}
\end{packed_item}

\noindent \underbar{\textbf{UC6 -Pregled pojedinačne karte}}
\begin{packed_item}
	
	\item \textbf{Glavni sudionik: }Registrirani korisnik
	\item  \textbf{Cilj:} Pregled kartice lokacije
	\item  \textbf{Sudionici:} Baza podataka
	\item  \textbf{Preduvjet:} Pregled svih sakupljenih karata
	\item  \textbf{Opis osnovnog tijeka:}
	
	\item[] \begin{packed_enum}
		
		\item Korisnik odabire jednu od kartica.
		\item Aplikacija prikazuje sve dostupne podatke o kartici lokacije.
	\end{packed_enum}
\end{packed_item}

\noindent \underbar{\textbf{UC7 -Pregled igrača u blizini}}
\begin{packed_item}
	
	\item \textbf{Glavni sudionik: }Registrirani korisnik
	\item  \textbf{Cilj:} Pregled igrača unutar radijusa od 50 km
	\item  \textbf{Sudionici:} Baza podataka
	\item  \textbf{Preduvjet:} Prijava
	\item  \textbf{Opis osnovnog tijeka:}
	
	\item[] \begin{packed_enum}
		
		\item Korisnik odabire opciju za pregled igrača u blizini.
		\item Aplikacija prikazuje popis svih igrača unutar radijusa od 50 km.
	\end{packed_enum}
\end{packed_item}

\noindent \underbar{\textbf{UC8 -Borba}}
\begin{packed_item}
	
	\item \textbf{Glavni sudionik: }Registrirani korisnik
	\item  \textbf{Cilj:} Odigrati borbu
	\item  \textbf{Sudionici:} Baza podataka, suigrač
	\item  \textbf{Preduvjet:} Traženje borbe
	\item  \textbf{Opis osnovnog tijeka:}
	
	\item[] \begin{packed_enum}
		
		\item Korisnik odabire opciju za pronalazak suparnika.
		\item Sustav sparuje korisnika s nekime od igrača.
		\item Korisnik izabire karte s kojima će igrati.
		\item Igrači su naizmjence na potezu, te na vlastitom potezu odabiru atribut kojime se igra i kartu koju žele.
		\item Nakon što su kartice potrošene, igra se završava.
	\end{packed_enum}
	
	\item  \textbf{Opis mogućih odstupanja:}
	
	\item[] \begin{packed_item}
		
		\item[1.a] Igrač odustane ili izgubi konenkciju
		\item[] \begin{packed_enum}
			
			\item  Sustav čeka korisnika 30sekundi kako bi se povezao nazad u sustav, ako se to ne dogodi, drugom igraču je dodijeljena pobjeda
			
		\end{packed_enum}
	\item[2.a] Igrač nema dovoljno karata za borbu
	\item[] \begin{packed_enum}
		
		\item  Prilikom pritiska na gumb "Traži borbu" sustav igraču javlja kako nema dovoljno karata za početak borbe (ili ih nema dovoljno ili su na coolodwnu) te mu javlja da pođe u istraživanje kako bi sakupio više karata te uvijek bio spreman i u mogućnosti na borbu
		
	\end{packed_enum}
		
	\end{packed_item}
\end{packed_item}

\noindent \underbar{\textbf{UC8.1 -Odabir karata za borbu}}
\begin{packed_item}
	
	\item \textbf{Glavni sudionik: }Registrirani korisnik
	\item  \textbf{Cilj:} Izbor karata za borbu
	\item  \textbf{Sudionici:} Baza podataka
	\item  \textbf{Preduvjet:} Pronađen je suigrač
	\item  \textbf{Opis osnovnog tijeka:}
	
	\item[] \begin{packed_enum}
		
		\item Aplikacija prikazuje sve sakupljene karte korisnika.
		\item Korisnik odabire 5 karata za borbu. 
		\item Korisnik odabire opciju za početak borbe. 
	\end{packed_enum}
	
	\item  \textbf{Opis mogućih odstupanja:}
	
	\item[] \begin{packed_item}
		
		\item[2.a] Pokušaj odabira karte koja je na cooldownu
		\item[] \begin{packed_enum}
			
			\item  sustav javlja da je karta na cooldownu te da je potrebno odabrati drugu kartu za borbu.
			
		\end{packed_enum}
		
	\end{packed_item}
\end{packed_item}

\noindent \underbar{\textbf{UC8.2 - Kraj borbe}}
\begin{packed_item}
	
	\item \textbf{Glavni sudionik: }Registrirani korisnik
	\item  \textbf{Cilj:} Završetak borbe
	\item  \textbf{Sudionici:} Baza podataka
	\item  \textbf{Preduvjet:} Borba je odigrana
	\item  \textbf{Opis osnovnog tijeka:}
	
	\item[] \begin{packed_enum}
		
		\item Aplikacija prikaže poruku ovisno o pobjedi ili porazu.
		\item Statistika oba suigrača se osvježava u bazi podataka.
		\item Kartama koje su korištene se smanjuju bodovi.
		\item Sustav vraća korisnika na početnu stranicu.
	\end{packed_enum}
\end{packed_item}
	
	\noindent \underbar{\textbf{UC9 - Prijava nove lokacije}}
	\begin{packed_item}
		
		\item \textbf{Glavni sudionik: }Registrirani korisnik
		\item  \textbf{Cilj:} Dodavanje nove kartice lokacije u igru
		\item  \textbf{Sudionici:} Baza podataka, kartograf
		\item  \textbf{Preduvjet:} Dovoljno iskustvo u igri
		\item  \textbf{Opis osnovnog tijeka:}
		
		\item[] \begin{packed_enum}
			
			\item Korisnik odabire opciju za prijavu nove lokacije.
			\item Korisnik unosi potrebne podatke.
			\item Sustav šalje prijavu kartografu.
		\end{packed_enum}
	\end{packed_item}
	
				
					
				\subsubsection{Dijagrami obrazaca uporabe}
					
					\textit{Prikazati odnos aktora i obrazaca uporabe odgovarajućim UML dijagramom. Nije nužno nacrtati sve na jednom dijagramu. Modelirati po razinama apstrakcije i skupovima srodnih funkcionalnosti.}
				\eject		
				
			\subsection{Sekvencijski dijagrami}
				
				\textbf{\textit{dio 1. revizije}}\\
				
				\textit{Nacrtati sekvencijske dijagrame koji modeliraju najvažnije dijelove sustava (max. 4 dijagrama). Ukoliko postoji nedoumica oko odabira, razjasniti s asistentom. Uz svaki dijagram napisati detaljni opis dijagrama.}
				\eject
	
		\section{Ostali zahtjevi}
		
			\textbf{\textit{dio 1. revizije}}\\
		 
			 \textit{Nefunkcionalni zahtjevi i zahtjevi domene primjene dopunjuju funkcionalne zahtjeve. Oni opisuju \textbf{kako se sustav treba ponašati} i koja \textbf{ograničenja} treba poštivati (performanse, korisničko iskustvo, pouzdanost, standardi kvalitete, sigurnost...). Primjeri takvih zahtjeva u Vašem projektu mogu biti: podržani jezici korisničkog sučelja, vrijeme odziva, najveći mogući podržani broj korisnika, podržane web/mobilne platforme, razina zaštite (protokoli komunikacije, kriptiranje...)... Svaki takav zahtjev potrebno je navesti u jednoj ili dvije rečenice.}
			 
			 
			 
	