\chapter{Opis projektnog zadatka}
		
		\textbf{\textit{dio 1. revizije}}\\
		
		\textit{Na osnovi projektnog zadatka detaljno opisati korisničke zahtjeve. Što jasnije opisati cilj projektnog zadatka, razraditi problematiku zadatka, dodati nove aspekte problema i potencijalnih rješenja. Očekuje se minimalno 3, a poželjno 4-5 stranica opisa.	Teme koje treba dodatno razraditi u ovom poglavlju su:}
		\begin{packed_item}
			\item \textit{potencijalna korist ovog projekta}
			\item \textit{postojeća slična rješenja (istražiti i ukratko opisati razlike u odnosu na zadani zadatak). Dodajte slike koja predočavaju slična rješenja.}
			\item \textit{skup korisnika koji bi mogao biti zainteresiran za ostvareno rješenje.}
			\item \textit{mogućnost prilagodbe rješenja }
			\item \textit{opseg projektnog zadatka}
			\item \textit{moguće nadogradnje projektnog zadatka}
		\end{packed_item}
		
		\textit{Za pomoć pogledati reference navedene u poglavlju „Popis literature“, a po potrebi konzultirati sadržaj na internetu koji nudi dobre smjernice u tom pogledu.}\\
		
		\textnormal{Cilj projekta grupe PI je razviti programsku podršku za stvaranje web aplikacije "GeoFighter".  GeoFighter je igra u kojoj se igrači međusobno bore kartama koje su sakupili na različitim stvarnim lokacijama. Cilj igrača je postići što veći rank na globalnoj ljestvici dok je njihov zadatak obići što više  gradova, naselja, planina, umjetničkih instalacija, znamenitosti i slično te sakupiti iste. Lokacije su u obliku igračih karata. Svaka karta ima naziv,  opis, fotografiju i oznaku 3 atributa. Igrači s 10 ili više pobjeda imaju opciju prijaviti željenu lokacije pored koje se nalaze. U prijavi lokacije igrači su dužni navesti ime lokacije i opis, dok jačinu karte određuje kartograf. \\\\ \underline{Kartograf} je osoba koja je zadužena za nadopunjavanje baze podataka s lokacijama koje su igrači prijavili. On ih može odbiti, potvrditi, urediti ili označiti da je potrebna potvrda s terena. Ukoliko kartograf prihvati lokaciju bez izmjena, igrač koji je lokaciju  prijavio dobiva tu kartu.}\\
		
		\textnormal{\underline{Administrator} ima najveće ovlasti. Uz ovlasti igrača i kartografa administrator može vidjeti i uređivati popis svih korisnika i njihovih osobnih  podataka. On može dodijeliti  igračima i privremeno isključenje iz igre. Dodatno može uređivati i postojeće lokacije u igri.}
		
		\textnormal{Registriranom korisniku omogućena je prijava u sustav s postojećim računom, dok neregistrirani korisnici imaju mogućnost kreiranja novog računa.}
		
		\textnormal{Za kreiranje korisničkog računa igrača potrebno je:}
		\begin{packed_item}
			\item \textbf{korisničko ime}
			\item \textbf{fotografija}
			\item \textbf{email adresa}
			\item \textbf{lozinka}
		\end{packed_item}
		\textit{Nakon  popunjavanja  podataka,  na njegovu email adresu se šalje link kojim može potvrditi svoj račun.}
		
		\textnormal{Za kreiranje korisničkog računa kartografa potrebno je:}
		\begin{packed_item}
			\item \textbf{broj IBAN računa}
			\item \textbf{fotografija osobne iskaznice}
		\end{packed_item}
		\textit{Kartografa potvrđuje administrator.}
		
		\textbf{\underline{Nakon kreiranja računa igrača}}
		\begin{figure}[H]
			
			\includegraphics[scale=0.2]{slike/example} \\%veličina u odnosu na širinu linije
			\caption{WorkInProgress Primjer UI igrača}
			\label{fig:example} %label mora biti drugaciji za svaku sliku
		\end{figure}
	
	
		\textnormal{Igračima je dostupan pregled profila igrača, globalna statistika odigranih borbi i sakupljenih lokacija te poredak svih igrača prema elo sustavu. Prilikom pregleda profila nekog igrača, prikazuju se sve karte koje je igrač ikad sakupio, rang na globalnoj ljestvici i statistike vezane uz zadnjih 10 borbi s drugim igračima. Istražujući svoju okolinu igrači sakupljaju lokacije tako da im dođu na dovoljnu udaljenost (~1km). Lokacije se sakupljaju u obliku igraćih karata koje igrači koriste za međusobne bitke. Pritiskom na gumb ukriženih mačeva, igrač pristupa popisu igrača koji se nalaze u radijusu 50km od njega. Ukoliko igrač želi započeti bitku pritiskom na ime drugog igrača šalje zahtjev za borbu. Nakon 10 osvojenih borbi igraču se otvara opcija da sam pridonosi bazi podataka GeoFighter-a tako da prijavljuje lokacije sa slikom, nazivom i opisom te lokacije.}\\
	
		\textnormal{IDEJE DOBRODOŠLE Igrač može vidjeti popis ostalih aktivnih igrača koji se nalaze unutar 50km i s njima ući u borbu. Borba se sastoji od 5 rundi. Svaka karta ima 3 atributa u vrijednosti 0-100 (0-najslabije, 100-najjače). Borba započinje bacanjem novčića koji odlučuje koji će igrač imati prvi potez. Igrač koji je pobijedio bacanje novčića započinje bitku bacanjem karte i odabirom jednog od tri atributa s karte. Protivnik baca svoju kartu te se gleda čija karta ima jači zadani atribut. Igrač koji je pobijedio trenutnu rundu započinje iduću rundu sve dok se ne odigra svih 5 rundi. Rating se dodjeljuje prema ELO ratingu koji se koristi u šahu.}
		\textnormal{Nakon borbe, korištene karte se ne mogu koristiti 72 sata kako bi se potaknulo igrače na aktivan život i istraživanje.}\\
		
		\textnormal{Sučelje kartografa razlikuje se od sučelja igrača. Kartografu se prijave pokazuju na karti te ukoliko nije zadovoljan prijavom i/ili želi provjeriti danu lokaciju sustav preko vanjskog servisa OSRM(Open source routing machine) dohvati rješenje problema trgovačkog putnika i kartografu put prikaže na karti. Nakon analize prijave, kartograf odlučuje hoće li prijavu odbiti nepovratno, prihvatiti prijavu te dodijeliti 3 atributa lokaciji te ju tako pretvoriti u igraću kartu, prepraviti prijavu nakon provjere na terenu (promjena slike ili opisa) te dodijeliti 3 atributa lokaciji ili će označiti prijavu kao nepotpunu te uz komentar "vratiti" prijavu igraču na doradu. Ukoliko je kartograf prihvatio prijavu bez dorada igraču se dodjeljuje igraća karta te lokacije.}\\
		
		\textnormal{Sučelje administratora sastoji se od sučelja igrača, kartografa te sučelja administratora. Prebacivanjem (toggle button) sučelja administrator bira želi li igrati kao igrač, raditi kao kartograf ili želi obavljati posao administratora. Administrator može pritiskom na gumb vidjeti popis svih registriranih igrača i kartografa, vidjeti i mijenjati njihove osobne podatke. Administrator jedini može brisati i uređivati postojeće lokacije tj. igraće karte.}\\

		\textnormal{Ideja ovog projekta je potaknuti igrače na fizičku aktivnost i na bolje upoznavanje okruženja u kojem žive. Na zabavan način i kroz igru, igrači se potiču na izlazak iz svojih domova i provođenje slobodnog vrijeme hodajući. Nezdrav način igranja igara u foteljama i stolcima, zamijenila je lagan fizička aktivnost. Uz to igračima se daje mogućnost obići njima do sada nepoznate znamenitosti svog mjesta. Lokacije za dobivanje igraćih karata prolaze kroz strogu kontrolu te garantiraju točne i igračima zanimljive informacije. Igrači ujedno mogu i dati prijedlog za novu lokaciju i time sačuvati od zaborava priču koja ih veže za tu lokaciju. Na ovaj način svim posjetiteljima tog mjesta omogućeno je lagano, zabavno i efikasno upoznavanje svih čari mjesta u kojem se nalaze.}\\
	
		\eject
		
		\section{Primjeri u \LaTeX u}
		
		\textit{Ovo potpoglavlje izbrisati.}\\

		U nastavku se nalaze različiti primjeri kako koristiti osnovne funkcionalnosti \LaTeX a koje su potrebne za izradu dokumentacije. Za dodatnu pomoć obratiti se asistentu na projektu ili potražiti upute na sljedećim web sjedištima:
		\begin{itemize}
			\item Upute za izradu diplomskog rada u \LaTeX u - \url{https://www.fer.unizg.hr/_download/repository/LaTeX-upute.pdf}
			\item \LaTeX\ projekt - \url{https://www.latex-project.org/help/}
			\item StackExchange za Tex - \url{https://tex.stackexchange.com/}\\
		
		\end{itemize} 	


		
		\noindent \underbar{podcrtani tekst}, \textbf{podebljani tekst}, 	\textit{nagnuti tekst}\\
		\noindent \normalsize primjer \large primjer \Large primjer \LARGE {primjer} \huge {primjer} \Huge primjer \normalsize
				
		\begin{packed_item}
			
			\item  primjer
			\item  primjer
			\item  primjer
			\item[] \begin{packed_enum}
				\item primjer
				\item[] \begin{packed_enum}
					\item[1.a] primjer
					\item[b] primjer
				\end{packed_enum}
				\item primjer
			\end{packed_enum}
			
		\end{packed_item}
		
		\noindent primjer url-a: \url{https://www.fer.unizg.hr/predmet/proinz/projekt}
		
		\noindent posebni znakovi: \# \$ \% \& \{ \} \_ 
		$|$ $<$ $>$ 
		\^{} 
		\~{} 
		$\backslash$ 
		
		\begin{longtabu} to \textwidth {|X[8, l]|X[8, l]|X[16, l]|} %definicija širine tablice, širine stupaca i poravnanje
			
			%definicija naslova tablice
			\hline \multicolumn{3}{|c|}{\textbf{naslov unutar tablice}}	 \\[3pt] \hline
			\endfirsthead
			
			%definicija naslova tablice prilikom prijeloma
			\hline \multicolumn{3}{|c|}{\textbf{naslov unutar tablice}}	 \\[3pt] \hline
			\endhead
			
			\hline 
			\endlastfoot
			
			\rowcolor{LightGreen}IDKorisnik & INT	&  	Lorem ipsum dolor sit amet, consectetur adipiscing elit, sed do eiusmod  	\\ \hline
			korisnickoIme	& VARCHAR &   	\\ \hline 
			email & VARCHAR &   \\ \hline 
			ime & VARCHAR	&  		\\ \hline 
			\cellcolor{LightBlue} primjer	& VARCHAR &   	\\ \hline 
			
		\end{longtabu}
		

		\begin{table}[H]
			
			\begin{longtabu} to \textwidth {|X[8, l]|X[8, l]|X[16, l]|} 
				
				\hline 
				\endfirsthead
				
				\hline 
				\endhead
				
				\hline 
				\endlastfoot
				
				\rowcolor{LightGreen}IDKorisnik & INT	&  	Lorem ipsum dolor sit amet, consectetur adipiscing elit, sed do eiusmod  	\\ \hline
				korisnickoIme	& VARCHAR &   	\\ \hline 
				email & VARCHAR &   \\ \hline 
				ime & VARCHAR	&  		\\ \hline 
				\cellcolor{LightBlue} primjer	& VARCHAR &   	\\ \hline 
				
				
			\end{longtabu}
	
			\caption{\label{tab:referencatablica} Naslov ispod tablice.}
		\end{table}
		
		
		%unos slike
		\begin{figure}[H]
			\includegraphics[scale=0.4]{slike/aktivnost.PNG} %veličina slike u odnosu na originalnu datoteku i pozicija slike
			\centering
			\caption{Primjer slike s potpisom}
			\label{fig:promjene}
		\end{figure}
		
		\begin{figure}[H]
			\includegraphics[width=.9\linewidth]{slike/aktivnost.PNG} %veličina u odnosu na širinu linije
			\caption{Primjer slike s potpisom 2}
			\label{fig:promjene2} %label mora biti drugaciji za svaku sliku
		\end{figure}
		
		Referenciranje slike \ref{fig:promjene2} u tekstu.
		
		\eject
		
	