\chapter{Zaključak i budući rad}
		
		\textnormal{Zadatak našeg projekta bio je razviti web aplikaciju "Geofighter", interaktivnu igru u kojoj korisnici fizički posjećuju lokacije u Hrvatskoj. Igrači dolaze do znamenitosti, ustanova, penju se do planinarskih domova ili istražuju Nacionalne parkove te na svakom cilju imaju priliku sakupiti kartu lokacije. Svaka karta lokacije ima 3 atributa koji određuju njenu vrijednost i upravo ti atributi služe kao materijal za borbu među igračima. Igrači iz svoje zbirke karata lokacija odabiru 3 karte, povežu se s igračem po želji blizu njih i kreću u borbu. Igrač također može unijeti novu lokaciju koja, ako je odobrena od strane kartografa, postaje pristupna svim igračima. Korisnik može imati i ulogu kartografa ili administratora koji imaju dodatne ovlasti.}\\
		 
		 \textnormal{Prva faza projekta bila je detaljno opisati projektni zadatak na čemu je grupa složno radila kako bi svi dijelovi zadatka bili razrađeni i svi članovi grupe detaljno upućeni u posao koji će se obavljati. Nakon razrađenog opisa, također složno napisali smo funkcionalne i ostale zahtjeve po kojima smo se kasnije orijentirali u raspoređivanju posla te pri oblikovanju i opisu arhitekture. Grupa se upoznala međusobno, a komunikaciju smo odlučili izvesti preko platforme Slack gdje je voditelj grupe otvorio kanale za razne dijelove projekta te nam je tamo slao sve obavijesti. Sastanke smo odlučili obavljati online na tjednoj bazi preko platforme Zoom i Teams, a po potrebi imali smo i koji izvanredni sastanak. Svi smo se upoznali s Git-om (i GitLab-om) te Latex-om (i TeXstudio-m). Pred kraj prve faze počeli smo se upoznavati s tehnologijama Java Spring za backend i Angular za frontend te izradili kostur aplikacije. Upoznavanje s tehnologijama oduzelo nam je puno vremena i istraživanja materijala na internetu s obzirom na to da se prije s njima nismo susreli. U zadnja dva tjedna prve faze završili smo početni (ne tako mali) dio backenda te login i signup funkcionalnosti skupa s nekim osnovnim komponentama frontenda.}\\
		 
		 \textnormal{Na početku druge faze uslijedila su 2 tjedna neprestanog programiranja, istraživanja i raznih uspjelih i neuspjelih pokušaja implementacije određenih funkcionalnosti. Grupa se ravnomjerno raspodijelila tako da svaka osoba radi i backend i frontend za njoj dodijeljenu funkcionalnost. Problema je bilo, pogotovo zato što smo se svi prvi puta susreli s ovako velikim projektom koristeći ove (nama nove) tehnologije Java Spring i Angular, ali su svi timskim radom uspješno riješeni. Uz tehničke probleme vezane uz prethodno nepoznavanje tehnologija, važno bi bilo istaknuti da je dokumentacija za leaflet library bila loše napisana za kombinaciju leaflet-routing-machine i typescript, što je u konačnici riješeno uz puno ispravljanja pogrešaka i proučavanja izvora na internetu kao što je Stack Overflow. Nakon ta 2 tjedna 90\% funkcionalnosti je završeno, čime je stvorena alfa inačica projekta. Ostatak druge faze projekta proveli smo dovršavajući i popravljajući funkcionalnosti, radeći na samom fizičkom izgledu aplikacije (engl. user-experience) i testirajući određene funkcionalnosti. Riješili smo probleme koji su se pojavili, dovršavali dokumentaciju i pripremali prezentaciju projekta.}\\
		 
		 \textnormal{Na kraju projekta, cijeli tim vrlo je zadovoljan završnim proizvodom našeg rada. Svi funkcionalni zahtjevi uspješno su implementirani, a izgledom i načinom na koji aplikacija radi u potpunosti smo zadovoljni. Dakako, ostvarenje projekta bilo bi kvalitetnije i brže da su članovi grupe imali više iskustva s tehnologijama koje smo koristili, ali i iskustva u programskom inženjerstvu. Ovaj projekt donio nam je mnoga nova znanja ne samo u podučju raznih tehnologija nego i u području timskog rada te općenito rada na projektima ovakve vrste. To znanje zasigurno će nam koristiti u budućnosti i biti oslonac u daljnjem učenju i usavršavanju pri radu na timskim projektima.}\\
		
		\eject 
