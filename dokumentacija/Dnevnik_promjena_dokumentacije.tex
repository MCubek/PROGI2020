\chapter{Dnevnik promjena dokumentacije}
		
		\textbf{\textit{Kontinuirano osvježavanje}}\\
				
		
		\begin{longtabu} to \textwidth {|X[2, l]|X[13, l]|X[4, l]|X[3, l]|}
			\hline \multicolumn{1}{|l|}{\textbf{Rev.}}	& \multicolumn{1}{l|}{\textbf{Opis promjene/dodatka}} & \multicolumn{1}{|l|}{\textbf{Autori}} & \multicolumn{1}{l|}{\textbf{Datum}} \\[3pt] \hline
			\endfirsthead
			
			\hline \multicolumn{1}{|l|}{\textbf{Rev.}}	& \multicolumn{1}{l|}{\textbf{Opis promjene/dodatka}} & \multicolumn{1}{|l|}{\textbf{Autori}} & \multicolumn{1}{l|}{\textbf{Datum}} \\[3pt] \hline
			\endhead
			
			\hline 
			\endlastfoot
			
			0.1 & Napravljen predložak	& Čubek & 7.10.2020. 		\\[3pt] \hline
			0.2 & Dopunjen predložak & Markovinović & 13.10.2020. 		\\[3pt] \hline
			0.3 & Napisan okviran opis projekt. & Vrhovec & 13.10.2020. 		\\[3pt] \hline 
			0.3.1 & Napisana potencijalna korist & Hrastnik & 17.10.2020. 		\\[3pt] \hline
			0.3.2 & Napisana postojeća slična rješenja & Markovinović & 17.10.2020. 		\\[3pt] \hline
			0.3.3 & Napisan mogućnost prilagodbe rješenja & Papak & 17.10.2020. 		\\[3pt] \hline
			0.3.4 & Napisan Opseg projektnog zadatka & Knežević & 17.10.2020. 		\\[3pt] \hline
			0.3.5 & Napisane moguće nadogradnje & Čubek & 17.10.2020. 		\\[3pt] \hline
			0.4 & Spojena poglavlja 2. cjeline & Čubek & 18.10.2020. 		\\[3pt] \hline
			0.4.1 & Spell check 2. cjeline & Hrastnik & 18.10.2020. 		\\[3pt] \hline
			0.4.2 & Dorada 2. cjeline & Vrhovec & 23.10.2020. 		\\[3pt] \hline
			0.4.3 & 3.cjelina, funkc. zahtjevi, obrasci uporabe & Vrhovec & 23.10.2020. 		\\[3pt] \hline
			0.4.4 & Dijagram obrazaca uporabe & Knežević & 25.10.2020. 		\\[3pt] \hline
			0.4.4 & Dijagram obrazaca uporabe & Štracak & 25.10.2020. 		\\[3pt] \hline

			\textbf{2.0} & Template  & Template & 28.09.2013. \\[3pt] \hline 
			&  &  & \\[3pt] \hline
			
			
		\end{longtabu}
	
	
		\textit{Moraju postojati glavne revizije dokumenata 1.0 i 2.0 na kraju prvog i drugog ciklusa. Između tih revizija mogu postojati manje revizije već prema tome kako se dokument bude nadopunjavao. Očekuje se da nakon svake značajnije promjene (dodatka, izmjene, uklanjanja dijelova teksta i popratnih grafičkih sadržaja) dokumenta se to zabilježi kao revizija. Npr., revizije unutar prvog ciklusa će imati oznake 0.1, 0.2, …, 0.9, 0.10, 0.11.. sve do konačne revizije prvog ciklusa 1.0. U drugom ciklusu se nastavlja s revizijama 1.1, 1.2, itd.}